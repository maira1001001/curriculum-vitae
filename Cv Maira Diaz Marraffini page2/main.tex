\PassOptionsToPackage{dvipsnames}{xcolor}

\documentclass[9pt,a4paper]{altacv}

%% AltaCV uses the fontawesome and academicon fonts
%% and packages. 
%% See texdoc.net/pkg/fontawecome and http://texdoc.net/pkg/academicons for full list of symbols.
%% 
%% Compile with LuaLaTeX for best results. If you
%% want to use XeLaTeX, you may need to install
%% Academicons.ttf in your operating system's font 
%% folder.


% Change the page layout if you need to
\geometry{left=1cm,right=9cm,marginparwidth=6.8cm,marginparsep=1.2cm,top=1.25cm,bottom=1.25cm,footskip=2\baselineskip}

% Change the font if you want to.

% If using pdflatex:
\usepackage[T1]{fontenc}
\usepackage[utf8]{inputenc}
\usepackage[default]{lato}

% If using xelatex or lualatex:
% \setmainfont{Lato}

\definecolor{VerdeFluor}{HTML}{00ff8b}
\definecolor{Negro}{HTML}{000000}
\definecolor{GrisPrimario}{HTML}{454545}
\definecolor{GrisSecundario}{HTML}{777777}

\colorlet{heading}{VerdeFluor}
\colorlet{accent}{Negro}
\colorlet{emphasis}{GrisPrimario}
\colorlet{body}{GrisSecundario}

% Change the bullets for itemize and rating marker
% for \cvskill if you want to
\renewcommand{\itemmarker}{{\small\textbullet}}
\renewcommand{\ratingmarker}{\faCircle}
%% sample.bib contains your publications
\addbibresource{sample.bib}

%\usepackage[colorlinks]{hyperref}
\usepackage{hyperref}

%header y footer
\usepackage{fancyhdr}



\begin{document}

\cvsection[page1sidebar]{Software Projects}

\cvevent{\href{{https://play.google.com/store/apps/details?id=com.newbalance.anticounterfeit}}{New Balance Real Chain}}{Junior Advance Javascript Backend with NodeJs 10.6 and Cardano Blockchain}{April 2019 -- February 2020}{} 

"New Balance Anti-counterfeit Backend" first release:
The project gives a solution that secure the authentication of New Balance products through Cardano blockchain technology. Using NFC technology and Tangem cards to check if the product is legitim and claim the ownership. 

\begin{itemize}
\item  \href{https://medium.com/atix-labs/atix-and-cardano-team-up-with-new-balance-against-fraud-f56e62632225}{Atixlabs, Medium notice}
\item \href{https://www.coindesk.com/cardano-and-new-balance-will-team-up-to-stop-counterfeit-kicks}{New Balance and Cardano Blockchain, Coindesk notice}
\end{itemize}

\divider

\cvevent{\href{https://github.com/maira1001001/JUEGO-de-Razas-y-Pelajes-CEDICA}{CEDICA Mini Game}}{Final work for \href{https://docs.google.com/document/d/101Lq5HFT7GclE-JbPqtF9A6G6FUg4jL4mFhS6rSIKbw/edit}{"Software laboratory" subject at UNLP}}{February 2019}{}
APK developed in Android SDK
\divider

\cvevent{Wordpress libraries and Back Office administration}{ \href{http://finhub.info/}{Finhub Blockchain  y Cryptocurrencies website}}{February 2018 -- July 2018}{}

Project to guide potential Bitcoin users. The website gives advices about brokers and wallets, and shows the Top 100 Cryptocurrencies by Market Capitalization.
The website is a Wordpress Template and I developed some worpress libraries. Furthermore, a back office administration was developed in Ruby on Rails.

\divider

%\cvevent{\href{https://play.google.com/store/apps/details?id=com.remediar.app}{APK “Remediar UNLP”}}{Agrupación Estudiantil "Remediar UNLP"}{Octubre 2017 -- Noviembre 2017}{} APK desarrollada con el framework Ionic 

\divider

\cvevent{\href{https://agenda.unlp.edu.ar/}{Contact e-book of Institutions at National University of La Plata}}{\href{http://www.cespi.unlp.edu.ar/}{CeSPI UNLP}}{Dic. 2015 -- Sept. 2016}{}
\begin{itemize}
    \item Search engine developed in Ruby on Rails 4.2 and ElasticSeach which request the API, also developed in Ruby on Rails.
    \item Back office administration developed in Ruby on Rails 4.2 and PostgresSQL where the contact institutions are created and indexed lately by the search engine.
\end{itemize}

\divider

\cvevent{Web platform for submitting the college degree at UNLP} {\href{http://www.cespi.unlp.edu.ar/}{CeSPI UNLP}}{May 2015 -- December 2015}{}
I worked on a Ruby on Rails  web platform which was designed for submitting the college degree.

\divider

\cvevent{Extension Project Web Platform, Release 2}{\href{http://www.cespi.unlp.edu.ar/}{CeSPI UNLP}}{May 2015 -- December 2015}{}
I worked on a new release of UNLP social projects. It was developed in Ruby on Rails 5 and MySql.

\divider

\cvevent{\href{http://dihuen.unlp.edu.ar/}{Dihuen Search Engine}}{\href{http://www.cespi.unlp.edu.ar/}{CeSPI UNLP}}{September 2014 -- December 2015}{}
\begin{itemize}
    \item Back Office administration developed in Ruby on Rails 4.2. We used "Nokogiri" gem for parsing HTML and "Sidekiq" gem for background processing.
    \item Search engine developed in Ruby on Rails. We used "Elasticsearch" gem for indexing.
\end{itemize}

\faExternalLinkSquare{\href{https://www.eldia.com/nota/2015-12-16-veinte-bibliotecas-a-un-click-de-distancia}{eldia.com newspape notice}}

\textit{This pdf was designed with \LaTeX}

\end{document}


