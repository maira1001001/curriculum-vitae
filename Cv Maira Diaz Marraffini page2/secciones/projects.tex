\cvsection{Proyectos}

\cvevent{\href{https://github.com/maira1001001/JUEGO-de-Razas-y-Pelajes-CEDICA}{Juego de “Razas y Pelajes” CEDICA}}{Trabajo final para \href{https://docs.google.com/document/d/101Lq5HFT7GclE-JbPqtF9A6G6FUg4jL4mFhS6rSIKbw/edit}{"Laboratorio de Software" UNLP}}{Febrero 2019}{}
APK desarrollada con Android SDK para la cátedra “Laboratorio de Software”

\divider
\cvevent{\href{https://finhub.info/compara/criptomonedas/}{Sección web “Criptomonedas”}}{Finhub Blockchain  y Criptomonedas website}{Febrero 2018 -- Julio 2018}{}
Módulo desarrollado con Javascript que consume de la API de \href{https://coinmarketcap.com/}{Coinmarketcap} las primeras 50 criptomonedas y las visualiza en el website.

\divider

\cvevent{\href{https://finhub.info/compara/precio-bitcoin-en-argentina/
}{Sección web “Precio Bitcoin en Argentina”}}{Finhub Blockchain  y Criptomonedas website}{Febrero 2018 -- Julio 2018}{}
Visualiza los valores de compra/venta del Bitcoin de distintos brokers y sus variaciones según el índice “Finhub”.Se desarrolló un backend en Ruby on Rails que administra los valores del módulo.

\divider

\cvevent{\href{https://play.google.com/store/apps/details?id=com.remediar.app}{APK “Remediar UNLP”}}{Agrupación Estudiantil "Remediar UNLP"}{Octubre 2017 -- Noviembre 2017}{} APK desarrollada con el framework Ionic 

\divider

\cvevent{\href{http://dihuen.unlp.edu.ar}{Metabuscador Dihuen Frontend}}{CeSPI UNLP}{Sept. 2014 -- Dic. 2015}{}
Desarrollado en Ruby on Rails. Se realizan búsqueda consultado a la API generada por el backend y se procesan las queries/consultas con “Elasticsearch”.

\divider

\cvevent{\href{https://www.eldia.com/nota/2015-12-16-veinte-bibliotecas-a-un-click-de-distancia}{Metabuscador Dihuen Backend}}{CeSPI UNLP}{Sept. 2014 -- Dic. 2015}{}
Desarrollado en Ruby on Rails. Se parsea HTML con la gema “Nokogiri”, se realiza background processing con la gema “Sidekiq”, se almacena el contenido parseado con MySQL y se crea una interfaz de acceso con Ruby on Rails API.

\divider

\cvevent{Agenda de contactos UNLP Backend}{CeSPI UNLP}{Dic. 2015 -- Sept. 2016}{}
Desarrollado en Ruby on Rails. En el backend se cargan los contactos de todas las entidades de la UNLP, se almacenan con MySQL y se crea una API para consultar los datos.

\divider

\cvevent{\href{https://agenda.unlp.edu.ar/}{Agenda de contactos UNLP Frontend}}{CeSPI UNLP}{Dic. 2015 -- Sept. 2016}{}
Participante del proyecto. Desarrollado en Ruby on Rails. El buscador consulta a la API generada por el backend y procesa la query con Elasticsearch. 

\divider

\cvevent{Trámite de solicitud de título UNLP}{CeSPI UNLP}{Mayo 2015 -- Dic. 2015}{}
Participante proyecto  desarrollado con Ruby on Rails y MySQL. Se conecta con la API del Siu-Guaraní. El sistema modela el workflow de la solicitud del título del estudiante.

\divider

%\cvevent{Portabilidad numérica Nextel Backend}{SnoopConsulting S.R.L.}{Abril 2011 -- Abril 2012}{}
%Participé del desarrollo del proyecto, utilizando Java y el Framework GWT (Google Web Toolkit).
%\medskip